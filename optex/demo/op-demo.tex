%% This is part of the OpTeX project, see http://petr.olsak.net/optex

\fontfam[LMfonts]              % Default font family: Latin Modern

\margins/1 a5 (1,1,1,1.4)cm    % A5 paper + 1cm margins
\typosize[9/10.5]              % 9pt font / 10.5pt baselineskip
\parindent=10pt                % typesetting parameters
\hyperlinks \Blue\Blue         % active hyperlinks
\activettchar`                 % in-text verbatim by `...`
\everyintt={\Red}              % in-text verbatim Red
\enquotes                      % use \"text" for English quotation

\tit Demonstration

\nonum\notoc\sec Contents

\maketoc  % Table of Contents is auto-generated here

\sec Lists

The lists have to be surrounded by `\begitems`
and `\enditems` sequences.

\begitems
* First item.
* Second item.
  \begitems \style i
    * Nested item list,
    * numbered by roman numerals.
  \enditems
* Last item.
\enditems

\secc Title of\nl Subsection

The subsection text\dots

\sec References

There is a numbered equation.
The number is auto-generated by `\eqmark` sequence.
$$\label[my-eq]
  a^2 + b^2 = c^2  \eqmark
$$
We can refer to Equation~\ref[my-eq]
on page~\pgref[my-eq].
We can refer to Table~\ref[my-tab] in
Section~\ref[tab-sec] too. And Figure~\ref[my-pic]
is on page~\pgref[my-pic].

\sec Hyperlinks

You can refer to \url{http://petr.olsak.net} using `\url`.
Or use `\ulink` if the raw URL needs to be hidden:
\ulink[http://petr.olsak.net/optex]{\OpTeX/ page}.
The parameter text is colorized and it becomes
an active link if the `\hyperlinks` sequence
is used at the beginning of the document.
Internal links are activated too.

\sec[tab-sec] Tables

The `\table` sequence can be used
instead of \"low level" `\halign`.
The following table is framed by the `\frame` sequence
to get a double frame.

\bigskip
\caption/t [my-tab]  Testing table.
\cskip
\centerline{%
   \frame{\table{|r|c|l|}{\crl
           \bf Title A & \bf Title B & \bf Title C \crll
           first & second & third \cr
           next  & text  & last  \crl }}}

\sec Images

The images (PDF, JPG, PNG, TIFF) can be inserted
by `\inspic` sequence. The `\caption`
can be added if you need to refer to a figure.

\label[my-pic]
\centerline {\picwidth=2.7cm \inspic{op-ring.png}}
\cskip
\caption/f The nonempty ideal of a simple ring --
           the ring itself.

\sec Verbatim

In-text verbatim is surrounded by the character declared
by `\activettchar` sequence. The listing can be surrounded
by `\begtt` and `\endtt` sequences
\begtt
This is verbatim.
  All characters are printed $$, \, # etc.
\endtt
or it can be included by `\verbinput` from an external file.

\verbinput (98-100) op-demo.tex

\sec Math

The Math alphabets
`\mit`, `\cal`, `\script`, `\frak`, `\bbchar`, `\bi`
are provided. For example:
$$
  {\bi A} = \pmatrix {\cal C  & \script C \cr
                      \frak M & \bbchar R }.
$$
Hundreds of AMS symbols are available:
$\sphericalangle, \boxplus, \Cup, \Cap, \ldots$

\sec Others

The `\fontfam` command selects a desired family of fonts.
The `\typosize` or `\typoscale` sequences set the size
and baselineskip of used fonts (including math fonts).
The `\fnote` generates a footnote\fnote{Like this} and
`\mnote` generates a margin note.
The `\margins` sets margins and paper dimensions.
The `\cite` sequence can be used for bibliographic citations.
The `\bib` sequence creates one bibliography record. Or
you can use `\usebib` for direct access to the {\tt.bib} files.
The list of features does not end here\dots

\bye
