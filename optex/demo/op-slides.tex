%% This is part of the OpTeX project, see http://petr.olsak.net/optex

\slides   % OpTeX slides style activated

\catcode`<=13                 % special printing of <text>
\def<#1>{\hbox{$\langle$\it#1\/$\rangle$}}
\everyintt={\catcode`\<=13}

\hyperlinks\Blue\Blue            % hyperlinks are used in the title page
\backgroundpic{op-slides-bg.png} % background picture

\verbchar`                    % inline verbatim
\enquotes                     % English quotes \"..."

\slideshow %------------------------------------------------------------


\tit \OpTeX/ slides

\subtit Petr Olšák\nl petr@olsak.net

\subtit\rm \url{http://petr.olsak.net/optex}

\pg; %------------------------------------------------------------------

\sec Basics

* A simple document looks like:

\begtt
\slides       % style initialized
%\wideformat  % 16:9
\slideshow    % partially uncovering ideas

\sec First slide
text
\pg;

\sec Second slide
text
\pg.
\endtt

* If `\slideshow` is missing or commented out then
  \"partially uncovering ideas" (see later) are deactivated. It is useful
  for printing.
* The `\slideshow` must be the last command in the declaration part of the
  document.
* By default, the slides have A5 landscape format.
  You can declare `\wideformat`. Then the height is the same
  but width is 263\,mm, i.\,e.~the ratio width:height is 16:9.

\pg; %------------------------------------------------------------------

\sec Title slide

* Title of the document (used at the first slide) is created
  by `\tit Title` (terminated by end of line).
* The `\subtit Author name etc.`\ (terminated by end of line)
  can be used after `\tit` at the first slide.
* You can use `\nl` for a new line in paragraphs or titles.

\sec Default design

* The paragraph texts are ragged right.
* Titles, subtitles, and section titles are centered.
* The `\sec` and `\secc` are printed without numbers.
* No paragraph indentation, a little vertical space between paragraphs.
* The Heros font family (aka Helvetica) is initialized as default.\nl
  Sans-serif FiraMath font for math typesetting is used.\nl
  If `\fontfam[<famname>]` precedes `\slides` then it rewrites default.
* The items in lists are started by a blue square (`\type X` and `\type x` ).

\pg; %------------------------------------------------------------------

\sec One slide (one page)

* Top-level item list is activated by default. The asterisk `*` opens
  new item at the top-level list.
* Nested items lists (second and more level) should be created in
  the `\begitems`\dots`\enditems` environments.
* Each slide (page) must be terminated by `\pg;` command.
* The last slide must be terminated by `\pg.` command or by `\bye`.

\begtt
\sec My ideas

* First idea
* Second idea
  \begitems
  * First sub-idea
  * Second sub-idea
  \enditems
* Final idea
\endtt

* Note: each page is processed in a group, so: put your own definitions
  (if~exist) before `\slideshow` or use `\global` assignment.

\pg; %------------------------------------------------------------------

\sec Partially uncovering ideas

* The control sequence `\pg` must be followed by:\pg+
\begitems
* the character `;` -- normal next page,\pg+
* the character `.` -- the end of the document,\pg+
* the character `+` -- next page keeps the same text
  and a next text is added (usable for partially uncovering of ideas).\pg+
\enditems

* Summary:
\begtt
\pg;    ... next page
\pg.    ... the end of the document
\pg+    ... uncover next text on the same page
\endtt
\pg+

* When `\slideshow` is not declared then `\pg+` is deactivated.\pg+
* The `\pg+` creates a new \"virtual page", so the current paragraph is
  terminated.

\pg; %------------------------------------------------------------------

\sec Example with partially uncovering ideas

The previous page was created by:

\begtt
\sec Partially uncovering ideas

* The control sequence `\pg` must be followed by:\pg+

\begitems
* the character `.` -- normal next page,\pg+
* the character `;` -- the end of the document,\pg+
* the character `+` -- next ... \pg+
\enditems

* Summary
...
* When `\slideshow` is not declared
  then `\pg+` is deactivated.\pg+
* The `\pg+` creates a new \"virtual page",
  so the current paragraph is terminated.
\pg;
\endtt

\pg; %------------------------------------------------------------------

\sec Notes to `\slideshow`

* When `\slideshow` is active then references created by `\ref`
  point to the first uncovering \"virtual" page where the destination is
  and references created by `\pgref` point to the last \"virtual" page.
* If the text overfulls the page (slide) then it follows to the next page without
  saying explicitly `\pg;`. But `\slideshow` cannot work in this case.\pg+
* If `\slideshow` then each part of page between two `\pg`'s or between
  `\slideshow` and the first `\pg` is processed in a local group.
* If not `\slideshow` then the document is not separated to groups.
  This can create different results. So, you can put `\slideopen` command
  instead of `\slideshow`. Then local groups are opened exactly as when
  `\slideshow` is used but `\slideshow` is not activated. Example:

\begtt
\slides
\def\foo...{...} % global definitions.
%\slideshow
\slideopen       % opens group for first page.
... first page
\pg;             % closes group and opens group for second page.
... second page
\pg.             % closes group of the last page.
\endtt

\pg; %------------------------------------------------------------------

\sec More about design

* You can use `\backgroundpic{<image-file>}` for putting an image to the background.
* You can re-declare `\footline` or re-define internal macros for design as
  you wish.
* The TeXGyre Heros font is used as default text font, the FiraMath is used
  for math.
* If you want to use another text font fmaily, use `\fontfam`
  {\em before} `\slides` command.
* If you want to use different math font, use
  `\loadmath{[font]}` before `\fontfam` (if used) and before `\slides`. For example:

\begtt
\loadmath{[Asana-Math]} % Math font: Asana
\fontfam[Termes]        % Text font: Termes
\slides
...
\endtt

* You can put the images or text wherever using `\putimege` or `\puttext`
  macros...

\pg; %------------------------------------------------------------------

\sec Putting images and texts wherever

* `\puttext <right> <up> {<text>}` puts a <text> to the desired place:
  It moves the current point <right> and <up>, puts the <text> and returns
  back, so the typesetting continues from previous position. The parameters
  <right> and <up> are dimensions. For example

\begtt
\puttext 0mm 50mm {\Red HELLO}
\endtt

  \puttext 0mm 50mm {\Red HELLO}
  prints red HELLO, as shown here.\pg+

* `\putpic <right> <up> <width> <height> {<image-file>}`
  puts the image with desired <width> and <height> at the position like
  `\puttext` puts the text.\pg+

  \putpic .8\hsize 20mm 30mm \nospec {op-ring.png}
* The ring above is the result of

\begtt
\putpic .8\hsize 20mm 30mm \nospec {op-ring.png}
\endtt
  %
  used at beginning of this paragraph.\pg+
* Use `\nospec` for <width> or <height> of the image if you don't want to specify both
  dimensions (because you don't want to change the image aspect ratio).

\pg; %------------------------------------------------------------------

\sec Limits of the \code{\\pg+} sequence

* The \code{\\pg+} sequence (partially uncovering ideas) cannot be used inside
  a~group.
* The exception is the nested environment `\begitems...\enditems`.
* The `\pg+` always finalizes the current paragraph.
  It is impossible to hide only a part of the horizontal mode.\pg+

\sec The \code{\\layers}\,...\code{\\endlayers} environment

If you really need something unsupported by `\pg+` then you can use

\begtt \catcode`\<=13
\layers <number>
<layered text>
\endlayers
\endtt

* The `\layers` opens <number> following pages with the same
  surrounding text. The counter `\layernum` is incremented from one to
  <number> . The <layered text> should use `\layernum` including
  conditions like `\ifnum\layernum` or `\ifcase\layernum`.
  See next page...

\pg; %------------------------------------------------------------------

\sec Example of \code{\\layers} environment

The `\slides` style provides a shortcut `\use` and a macro `\pshow` (means
partially show):

\begtt
\def\use#1#2{\ifnum\layernum#1\relax#2\fi}
\def\pshow#1{\use{=#1}\Red \use{<#1}\Transparent \ignorespaces}
\endtt
`\use{=<num>}{<something>}` does <something> only if `\layernum=<num>`.\kern-1em

The `{\pshow<num> <text>}` prints <text> in Red when current layer is equal
to <num> or it prints <text> normally when the current layer is greater than <num>.
The transparent (invisible) text is used in other cases.

The following dance:
\layers 3
{\pshow2 Second text.} {\pshow3 Third text.} {\pshow1 First text.}
\endlayers
\pg+

was generated by

\begtt
\layers 3
{\pshow2 Second text.} {\pshow3 Third text.} {\pshow1 First text.}
\endlayers
\endtt

\pg+
* The <layered text> is treated as a macro parameter. So, you cannot use verbatim
  nor `\sec` titles here. Maximal one `\layers` environment can be per one
  page (terminated by `\pg+` or `\pg;` or `\pg.`).

\pg; %------------------------------------------------------------------

\sec Comparison \OpTeX/ slides with Beamer\fnote{\url{http://www.ctan.org/pkg/beamer}}

The \LaTeX{} package {\bf\Blue Beamer} gives much more features and many themes
are prepared for Beamer, {\bf\Red but}
\pg+
* the user of Beamer is forced to {\em program} his/her document using
  dozens of \code{\\begin{foo}} and \code{\\end{foo}} and many other
  programming constructions,\pg+
* plain \TeX{} gives you a possibility to simply
  {\em write} your document with minimal markup. The result is more compact.
  You can concentrate on the contents of your document, not on the
  programming syntax.\pg+
* User needs to read 250 pages of doc for understanding Beamer,\pg+
* on the other hand, you need to read only eleven
  slides\fnote{this twelfth slide isn't counted}
  and you are ready to use {\bf\Blue\OpTeX/ slides}.

\pg; %------------------------------------------------------------------

\null
\vskip2cm
\centerline{\typosize[35/40]\bf Thanks for your attention}\pg+

\vskip2cm
\centerline{\Blue\typosize[60/70]\bf Questions?}

\pg. %----------------------------- THE END ----------------------------

