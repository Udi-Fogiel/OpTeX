
\fontfam[LMfonts]
\typosize[11/13]
\enlang
\margins/1 a4 (2,2,2,2)cm

\catcode`<=13
\def<#1>{\hbox{$\langle$\it#1\/$\rangle$}}
\everyintt={\catcode`\<=13}
\everytt=\everyintt
\chardef\"="201C

\def\new{\mnote{\Red$\blacktriangleleft$}}
\fixmnotes\right

\activettchar`

\hyperlinks{\Blue}{\Green}
\insertoutline{CONTENTS} \outlines{0} 


\tit \OpTeX/\nl Format Based on Plain \TeX/ and OPmac\fnotemark1
%%%%%%%%%%%%%%%%%%%%%%%%%%%%%%%%%%%%%%%%%%%%%%%%%%%%%%%%

\hfill Version 0.06

\centerline{\it Petr Olšák, 2020}

\bigskip
\centerline{\url{http://petr.olsak.net/optex}}

\fnotetext {The OPmac package is a set of simple additional macros to plain\TeX{}. It
enables users to take advantage of basic \LaTeX/ functionality but keeps
plain \TeX/ simplicity. See
\url{http://petr.olsak.net/opmac-e.html} for more information about it.
For OPmac users: the red triangle {\Red$\blacktriangleleft$} in the right margin
means that there is a difference from standard OPmac features.}

\notoc\nonum \sec Contents
\maketoc

\nonum \sec Introduction
%%%%%%%%%%%%

\OpTeX/ is \LuaTeX/ format with plain \TeX/ and OPmac. Only \LuaTeX/ engine
is supported. The main goal of \OpTeX/ is:

\begitems
* \OpTeX/ keeps the simplicity (like in plain \TeX/ and OPmac macros).
* There is no old obscurities concerning with various 8-bit encodings and
  various engines.
* \OpTeX/ provides a powerful font selection system (for Unicode font
  families, of course).
* \OpTeX/ supports hyphenations of all languages installed in your \TeX/ system.
* All features from OPmac macros are copied.
* Macros are documented in the same place where code is (macros for printing
  this documentation will come in the future).
* User name space of control sequences is separated from internal name space
  of OpTeX and primitives (`\foo` versus `\_foo`).
\enditems

\OpTeX/ should be a modern plain \TeX/ with power from OPmac (fonts selection
system, colors, external graphics, references, hyperlinks, indexing,
bibliography, ...) with preferred Unicode fonts.

If you need to customize your document or you need to use something
very specific, then you can copy relevant parts of \OpTeX/ macros into your macro
file and do changes of these macros here. This is significant difference from
\LaTeX/ or ConTeXt, which are an attempt to create a new user level with a
plenty of non-primitive parameters and syntax hidding \TeX/ internals.
The macros from \OpTeX/ are simple and straightforward because they solves only
what is excplicitly needed, they does not create a new user level over \TeX/.
And you can use them, understand them an modify them.

\OpTeX/ offers a markup language for authors of texts (like \LaTeX),
i.e. the fixed set of tags to define the structure of the document. This
markup is different from the \LaTeX{} markup. It may offer to write the
source text of the document somewhat clearer and more attractive. 

\new 
{\bf Disclaimer:} This software is under construction. It is possible
that some features documented here will be changed in future. There exists a
large part of original OPmac macros included in \OpTeX/ which waits to its
re-implementation at current state of development.


\sec Starting with \OpTeX/
%%%%%%%%%%%%%%%%%%%%%%%%%

\new
\OpTeX/ is compiled as a format for \LuaTeX/. Maybe there is a command
`optex` in your \TeX/ distribution. Then you can write into command line

\begtt
optex document
\endtt
%
You can try to process `optex op-demo` or `optex optex-doc`.

If there is no `optex` command, see more information about installation
\OpTeX/ at \url{http://petr.olsak.net/optex}. 

A minimal document should be

\begtt
\fontfam[LMfonts]
Hello World! \bye
\endtt

The first line `\fontfam[LMfonts]` tells that Unicode Latin Modern 
fonts (derived from Computer Modern) are used. If you omit this line then
preloaded Latin Modern fonts are used but preloaded fonts cannot be in
Unicode\fnote
{This is technical limitations of \LuaTeX/ for fonts downloaded in formats:
only 8bit fonts can be preloaded.}.
So the sentence `Hello World` will be OK without the first line, but you 
cannot print such sentence in another languages (for example `Ahoj světe!`) 
where Unicode fonts are needed
because of the characters like `ě` are not mapped correctly in preloaded
fonts.

A somewhat larger example with common settings should be:

\begtt
\fontfam[Termes]   % selecting Unicode font family Termes
\typosize[11/13]   % setting the basic font size and the baselineskip
\margins/1 a4 (1,1,1,1)in   % setting 1in margins for A4 paper
\cslang            % Czech hyphenation patterns

Tady je text.
\bye
\endtt
%
You can look at `op-demo.tex` file for more examples.


\sec Page layout
%%%%%%%%%%%%%%%%

\secc Setting the margins
%%%%%%%%%%%%%%%%%%%%%%%%

\OpTeX/ declares paper formats a4, a4l (landscape~a4), a5, a5l, b5, letter and
user can declare another own format by `\sdef`:

\begtt
\sdef{pgs:b5l}{(250,176)mm} 
\sdef{pgs:letterl}{(11,8.5)in}
\endtt

The `\margins` command declares margins of the document. This command have
the following parameters:

\begtt
\margins/<pg> <fmt> (<left>,<right>,<top>,<bot>)<unit>
  example:
\margins/1 a4 (2.5,2.5,2,2)cm
\endtt

Parameters are:

\begitems
* <pg> \dots\ `1` or `2` specifies one-page or two-pages design.
* <fmt> \dots\ paper format (a4, a4l, a5, letter, etc. or user defined).
* <left>, <right>, <top>, <bot> \dots\ gives the amount of left, right,
      top and bottom margins.
* <unit> \dots\ unit used for values <left>, <right>, <top>, <bot>.
\enditems

Each of the parameters <left>, <right>, <top>, <bot> can be empty.
If both <left> and <right> are nonempty then `\hsize` is set. Else `\hsize`
is unchanged. If both <left> and <right> are empty then typesetting area is
centered in the paper format. The analogical rule works when <top> or <bot>
parameter is empty (`\vsize` instead `\hsize` is used). Examples:

\begtt
\margins/1 a4 (,,,)mm   % \hsize, \vsize untouched, 
                        % typesetting area centered
\margins/1 a4 (,2,,)cm  % right margin set to 2cm
                        % \hsize, \vsize untouched, vertically centered
\endtt

If `<pg>=1` then all pages have the same margins. If `<pg>=2` then the
declared margins are true for odd pages. The margins at the even pages are
mirrored in such case, it means that <left> is replaced by <right> and vice
versa.

The `<fmt>` can be in the form `(<width>,<height>)<unit>` where `<unit>` is
optional. If it is missing then `<unit>` after margins specification is
used. For example:

\begtt
\margins/1 (100,200) (7,7,7,7)mm
\endtt
%
declares the paper 100$\times$200\,mm with all four margins 7\,mm. The spaces
before and after <fmt> parameter are necessary.

The command `\magscale[<factor>]` scales the whole typesetting area. 
\new The fixed point of such scaling is the upper left corner of the paper sheet. 
Typesetting (breakpoints etc.) is unchanged. All units are relative after
such scaling. Only paper formats dimensions stays unscaled. Example:

\begtt
\margins/2 a5 (22,17,19,21)mm
\magscale[1414] \margins/1 a4 (,,,)mm
\endtt
%
The first line sets the `\hsize` and `\vsize` and margins for final
printing at a5 format. The setting on the second line centers the scaled 
typesetting area to the true a4 paper while breaking points for paragraphs
and pages are unchanged. It may be usable for 
review printing. After review is done, the second line can be commented out.

\secc Concept of default page
%%%%%%%%%%%%%%%%%%%%%%%%%%%%%

\OpTeX/ uses for page design very similar \"output routine" like plain
\TeX/. There is `\headline` followed by \"page body" followed by
`\footline`. The `\headline` is empty by default and it can be used
for running headers repeated on each page. The `\footline` prints
page number in the center by default. You can set the `\footline` to empty
using `\nopagenumbers` macro.

The margins declared by `\margins` macro is concerned to the page body,
i.e.\ the `\headline` and `\fooline` are placed to the top and bottom
margins.

The distance between `\headline` and the top of page body is given by
`\hedalinedist` register. The distance between bottom of page body and `\footline` is
given by `\footlinedist`. The default values are:

\begtt
\headline = {}
\footline = {\_hss\_rmfixed \_folio \_hss} % \folio expands to page number
\headlinedist = 14pt % from baseline of \hedaline to top of page body 
\footlinedist = 24pt % from last line in pagebody to baseline of footline
\endtt

The page body should be divided to top insertions (floating tables and
figures), real text and footnotes. Typicaly only real text is here. 

The `\pgbackground` tokens list is empty by default but it ca be used for
creating background of each page (colors, picture, watermark for example).
The macro `\draft` uses this register and puts big text DRAFT as watermark
to each page. You can try it.

More about the page layout is documented in files `parameters.opm` and 
`output.opm`.

\secc Footnotes and marginal notes
%%%%%%%%%%%%%%%%%%%%%%%%%%%%%%%%%%

The plain \TeX/'s macro `\footnote` can be used as usual. But a new macro 
`\fnote{<text>}` is defined. The footnote mark is added automatically and it
is numbered 
\new
on each chapter from one\fnote
{You can declare `\fnotenumglobal` if you want footnotes numbered in whole
document from one or `\fnotenumpages` if you want footnotes numbered at each
page from one. Default setting is `\fnotenumchapters`}.
The <text> is scaled by `\typoscale[800]`. 
User can redefine fotnote mark or scaling, as shown in the file
`fnotes.opm`.

The `\fnote` macro is fully applicable only in \"normal outer" paragraph.
It doesn't work inside boxes (tables for example). If you are solving such
case you can use `\fnotemark<number>` inside the box (only the footnote mark is
generated). When the box is finished you can use `\fnotetext{<text>}`. This
macro puts the <text> to the footnote. The <number> after `\fnotemark`
have to be "1" if only one such command is in the box. Second `\fnotemark`
inside the same box have to have the parameter `2` etc. 
The same number of `\fnotetext`s have to be written 
after the box as the number of `\fnotemark`s inserted inside the box.

The marginal note can be printed by the `\mnote{<text>}` macro. The <text>
is placed to the right margin on the odd pages and it is placed to the left
margin on the even pages. This is done after second \TeX{} run because the
relevant information is stored in an external file. If you need to place the
notes only to the fixed margin write `\fixmnotes\right` or
`\fixmnotes\left`.

The <text> is formatted as a little paragraph with the maximal width
`\mnotesize` ragged left on the left margins or ragged right on the right
margins. The first line of this little paragraph is at the same height as
the invisible mark created by `\mnote` in the current paragraph. The
exceptions are possible by `\mnoteskip` register. You can implement such
exceptions to each `\mnote` manually in final printing in order to margin
notes do not overlap. The positive value of `\mnoteskip` shifts the note up
and negative value shifts it down. For example
`\mnoteskip=2\baselineskip \mnote{<text>}` shifts this (and only this) note 
two lines up.


\sec Fonts
%%%%%%%%%%

\secc Font families
%%%%%%%%%%%%%%%%%%%

You can select the font family by `\fontfam[<Family-name>]`.
The argument <Family-name> is case insensitive and spaces are ignored. So,
`\fontfam[LM Fonts]` is equal to `\fontfam[LMfonts]` and it is equal
to `\fontfam[lmfonts]`. Several aliases are prepared, thus 
`\fontfam[Latin Modern]` can be used for loading Latin Modern family too.

If you write `\fontfam[?]` then all font families registered in \OpTeX/ 
are listed on the terminal and in the log file.

If you write `\fontfam[catalog]` then a catalog of all fonts registered in
\OpTeX/ and available in your \TeX/ system is printed. And the instructions
how to register your own font family is appended in such catalog.

If the family is loaded then {\em font modifiers} applicable in such font family
are listed on the terminal: (`\caps`, `\cond` for example).
And there are four basic {\em variant selectors} (`\rm`, `\bf`, `\it`, `\bi`).
The font modifiers (`\caps`, `\cond` for example) can
be used immediately before a basic variant selector and they
can be (independently) combined: `\caps\it` or `\cond\caps\bf`. The
modifiers keeps their internal setting until group ends or until another
modifier which negates the previous feature is used. So
`\caps \rm text \it text`  uses normal and italics in Caps and SmallCaps.

\new
There is one special variant selector `\currvar` which does not change the
selected variant but reloads the font in respect of the (maybe newly
specified) font modifiers(s).

The context between variants `\rm`--`\it` and `\bf`--`\bi` is kept by the `\em`
macro (emphasize text).  
It switches from current `\rm` to `\it`, from current `\it` to `\rm`, from
current `\bf` to `\bi` and from current `\bi` to `\bf`.
The needed italics correction `\/` is inserted automatically. Example:

\begtt
This is {\em important} text.     % = This is {\it important\/} text.
\it This is {\em important} text. % = This is\/ {\rm important} text.
\bf This is {\em important} text. % = This is {\bi important\/} text.
\bi This is {\em important} text. % = This is\/ {\bf important} text.
\endtt

\new
More about the \OpTeX/ font selection system is written the file
`fonts-select.opm`. You can mix more font families in your document, you can
declare your variant selectors or modifiers etc.

\secc Font sizes
%%%%%%%%%%%%%%%

The command `\typosize[<fontsize>/<baselineskip>]` sets the font size of text and
math fonts and baselineskip. If one of these two parameters is empty, the
corresponding feature stays unchanged. Don't write the unit of these
parameters. The unit is internally set to `\ptunit` which is 1pt by default.
You can change the unit by the command `\ptunit=<something-else>`, 
for instance `\ptunit=1mm` enlarges all font sizes declared by `\typosize`.
Examples:

\begtt
\typosize[10/12]   % default of plainTeX
\typosize[11/12.5] % font 11pt, baseline 12.5pt
\typosize[8/]      % font 8pt, baseline unchanged
\endtt

The commands for font size setting described in this section
have local validity. If you put them into a group, 
the settings are lost when the group is finished. If you set 
something relevant with paragraph shape (baselineskip given by 
`\typosize` for example) then you must first finalize the 
paragraph and second to close the group: 
`{\typosize[12/14] ...<text of paragraph>... \par}`. 

The command
`\typoscale[<font-factor>/<baselineskip-factor>]`
sets the text and math fonts
size and baselineskip as a multiple of the current fonts size and
baselineskip. The factor is written in \"scaled"-like way, it means that 1000
means factor one. The empty parameter is equal to the parameter 1000,
i.e. the value stays unchanged. Examples:

\begtt
\typoscale[800/800]    % fonts and baselineskip re-size to 80 %
\typoscale[\magstep2/] % fonts bigger 1,44times
\endtt

First usage of `\typosize` or `\typoscale` macro in your document sets so
called {\em main values}, i.~e. main font size and main baselineskip. They are internally 
saved in registers `\mainfosize` and `\mainbaselineskip`. 

\new
The `\typoscale` comand does scaling in respect to current values by default. 
If you want to do it in respect to main values, type `\scalemain` immediately
before `\typoscale` command.

\begtt
\typosize[12/14.4] % first usage in document, sets main values internally
\typosize[15/18]   % bigger font
\scalemain \typoscale[800/800] % reduces from main values, no from current.
\endtt

The size of the current font can be changed by the command
`\thefontsize[<font-size>]` or can be rescaled by
`\thefontscale[<factor>]`. These macros don't change math fonts sizes nor
baselineskip.

\new
There is `\setfontsize{<size-spec>}` command which behaves like
font modifiers and sets given font size o fonts loaded by next variant selectors. 
For example `\setfontsize{at15pt}\currvar`.

More information about resizing of fonts is documented in `fonts-resize.opm`
file. 

\secc Typesetting math
%%%%%%%%%%%%%%%%%%%%%

\OpTeX/ preloads a collection of 7bit Computer Modern and AMS fonts.
You can use them in any size and in the `\boldmath` variant. 
%
\new 
Most declared font families are configured with recommended Unicode
math font. This font is automaticlally loaded unless you specify
`\noloadmath` before first `\fontfam` command. See log file for more
information about loading text font family and Unicode math fonts. If you
prefer another Unicode math font, specify it by `\loadmath{[<font-file>]}`
or `\loadmath{font-name}` before first `\loadfam` command.

Hundreds math symbols and operators like in AMS\TeX/ are accesible. 
For example  `\alpha` $\alpha$, `\geq`~$\geq$, `\sum` $\sum$, 
`\sphericalangle` $\sphericalangle$, `\bumpeq`, $\bumpeq$. See AMS\TeX/
manual for complete list of symbols.

The following math alphabets are available:

\begtt    \catcode`\$=3 \catcode`/=0 \medmuskip=0mu \adef{ }{ }
\mit     % mathematical variables    $abc-xyz,ABC-XYZ$
\it      % text italics              $/it abc-xyz,ABC-XYZ$
\rm      % text roman                $/rm abc-xyz,ABC-XYZ$
\cal     % normal calligraphics      $/cal ABC-XYZ$
\script  % script                    $/script ABC-XYZ$
\frak    % fracture                  $/frak abc-xyz,ABC-XYZ$
\bbchar  % double stroked letters    $/bbchar ABC-XYZ$
\bf      % sans serif bold           $/bf abc-xyz,ABC-XYZ$
\bi      % sans serif bold slanted   $/bi abc-xyz,ABC-XYZ$
\endtt

The last two selectors `\bf` and `\bi` select the sans serif fonts regardless
current text font family. 

The math fonts can be scaled by `\typosize` and `\typoscale` macros.
Two math fonts collections are prepared: `\normalmath` for normal weight
and `\boldmath` for bold. The first one is set by default.

\new
You can use `\mathbox{<text>}` inside math mode. It behaves as `{\hbox{<text>}}`
(i.e. the <text> is printed in horizontal non-math mode)
but the size of the <text> is adapted to the context of math size (text or script or
scriptscript). 


\sec Typical elements of document
%%%%%%%%%%%%%%%%%%%%%%%%%%%%%%%%%

\secc[chap] Chapters and sections
%%%%%%%%%%%%%%%%%%%%%%%%%%

The document can be divided into chapters, sections and subsections and titled
by `\tit` command. The parameters are separated by the end of current line (no
braces are used):

\begtt
\tit Document title <end of line>
\chap Chapter title <end of line>
\sec Section title <end of line>
\secc Subsection title <end of line>
\endtt

The chapters are numbered by one number, sections by two numbers
(chapter.section) and subsections by three numbers. If there are no chapters
then section have only one number and subsection two.

The implicit design of the titles of chapter etc.\ are implemented in the
macros `\printchap`, `\printsec` and `\printsecc`. User can simply change
these macros if he/she needs another behavior.

The first paragraph after the title of chapter, section and subsection is
not indented but you can type `\let\firstnoindent=\relax` if you need all
paragraphs indented.

If a title is so long then it breaks to more lines. It is better to hint the
breakpoints because \TeX/ does not interpret the meaning of the title.
User can put the `\nl` (it means newline) macro to the breakpoints.

The chapter, section or subsection isn't numbered if the `\nonum` precedes.
And the chapter, section or subsection isn't delivered to the table of
contents if `\notoc` precedes.

\secc[cap] Another numbered objects
%%%%%%%%%%%%%%%%%%%%%%%%%%%%%

Apart from chapters, sections and subsections, there are another
automatically numbered objects: equations and captions for tables and
figures.

If user write the `\eqmark` as the last element of the display mode then
this equation is numbered. The format is one number in brackets. This number
is reset in each section. 

If the `\eqalignno` is used, then user can put `\eqmark` to the last column
before `\cr`. For example:

\begtt
\eqalignno{
    a^2+b^2 &= c^2 \cr
          c &= \sqrt{a^2+b^2} & \eqmark \cr}
\endtt

The next numbered object is caption which is tagged by `\caption/t` for
tables and `\caption/f` for figures. 
The caption text follows.
The `\cskip` can be used between `\caption` text and the real object (table
or figure). It is irrelevant, if caption or object is first.
The `\cskip` creates appropriate vertical space between them. Example:

\begtt
\caption/t The dependency of the computer-dependency on the age.
\cskip
\hfil\table{rl}{
    age   & value \crl\noalign{\smallskip}
    0--1  & unmeasured \cr 
    1--6  & observable \cr
    6--12 & significant \cr
   12--20 & extremal \cr
   20--40 & normal \cr
   40--60 & various \cr
   60--$\infty$ & moderate}
\endtt

This example produces:

\medskip
\caption/t The dependency of the computer-dependency on the age.
\cskip
\hfil\table{rl}{age   & value \crl\noalign{\smallskip}
                0--1  & unmeasured \cr 
                1--6  & observable \cr
                6--12 & significant \cr
               12--20 & extremal \cr
               20--40 & normal \cr
               40--60 & various \cr
               60--$\infty$ & moderate}
\medskip

You can see that the word \"Table" followed by a number is added by the macro 
`\caption/t`. 
The caption text is centered. If it occupies more lines then the 
last line is centered.

The macro `\caption/f` behaves like `\caption/t` but it is intended for
figure captions.
The word (Table, Figure) depends on the actual selected language (see
section~\ref[lang] about languages). 

If you wish to make the table or figure as floating object, you need to use
plain \TeX/ macros `\midinsert` or `\topinsert` terminated by `\endinsert`.

\secc References
%%%%%%%%%%%%%%%

Each automatically numbered object documented in
sections \ref[chap] and \ref[cap] can be referenced 
\new
if optional parameter
`[<label>]` is appended to `\chap`, `\sec`,
`\secc`, `\caption/t`, `\caption/f` or `\eqmark`. The alternative syntax is
to use `\label[<label>]` before mentioned commands (not necessarily directly
before). The reference is realized by `\ref[<label>]` or `\pgref[label]`.
Example:

\begtt
\sec[beatle] About Beatles

\hfil\table{rl}{...} % the table
\cskip
\caption/t [comp-depend] The dependency of the computer-dependency on the age.

\label[pythagoras]
$$ a^2 + b^2 = c^2 \eqmark $$

Now we can point to the section~\ref[beatle] on the page~\pgref[beatle] 
or write about the equation~\ref[pythagoras]. Finally there 
is an interesting Table~\ref[comp-depend].
\endtt

If there are forward referenced objects then user have to run \TeX{} twice.
During each pass, the working `*.ref` file (with references data) is created
and this file is used (if it exists) at the beginning of the document.

You can create a reference to whatever else by commands
`\label[<label>]\wlabel{<text>}`. The connection between <label> and
<text> is established. The `\ref[<label>]` will print <text>.

\secc Hyperlinks, outlines
%%%%%%%%%%%%%%%%%%%%%%%%%

If the command `\hyperlinks{<color-in>}{<color-out>}` is used at the beginning of
the file, then the following objects are hyperlinked when PDF output is used:

\begitems
* numbers generated by `\ref` or `\pgref`,
* numbers of chapters, sections and subsections in the table of contents,
* numbers or marks generated by `\cite` command (bibliography references),
* texts printed by `\url` command.
\enditems

The last object is an external link and it is colored by
`<color-out>`. Others links are internal and they are colored by
`<color-in>`. Example:

\begtt
\hyperlinks \Blue \Green % internal links blue, URLs green.
\endtt

You can use another marking of active links: by frames which are visible in
the PDF viewer but invisible when the document is printed. The way to do it
is to define the macros `\pgborder`, `\tocborder`, `\citeborder`,
`\refborder` and `\urlborder` as the triple of RGB components of the used
color. Example:

\begtt
\def\tocborder {1 0 0}  % links in table of contents: red frame
\def\pgborder {0 1 0}   % links to pages: green frame
\def\citeborder {0 0 1} % links to references: blue frame
\endtt

By default these macros are not defined. It means that no frames are created.

There are \"low level" commands to create the links. You can specify the
destination of the internal link by `\dest[<type>:<label>]`. The
active text linked to the `\dest` can be created by
`\ilink[<type>:<label>]{<text>}`. The `<type>` parameter is one of
the `toc`, `pg`, `cite`, `ref` or another special for your purpose. 
These commands create internal links only when `\hyperlinks` is decared.

The `\url` macro prints its parameter in `\tt` font and creates a potential
breakpoints in it (after slash or dot, for example). If `\hyperlinks`
declaration is used then the parameter of `\url` is treated as an external URL link.
An example: `\url{http://www.olsak.net}` creates \url{http://www.olsak.net}.
The characters \code{\%}, `\`, `#`, `{` and `}` have to be protected by
backslash in the `\url` argument, the other special characters `~`,
`^`, `&` can be written as single character\fnote
{More exactly, there is the same rules as for \code{\\code} command, see
section~\ref[verbatim].}.
You can insert the `\|` command 
in the `\url` argument as a potential breakpoint.

If the linked text have to be different than the URL, you can use
`\ulink[<url>]{text}` macro. For example:
`\ulink[http://petr.olsak.net/optex]{\OpTeX/ page}`
creates
\ulink[http://petr.olsak.net/optex]{\OpTeX/ page}.

The PDF format provides {\em outlines} which are notes placed in the special frame of
the PDF viewer. These notes can be managed as structured and hyperlinked
table of contents of the document. The command `\outlines{<level>}` creates
such outlines from data used for table of contents in the document. The
<level> parameter gives the level of opened sub-outlines
in the default view. The deeper levels can be open by mouse click on the
triangle symbol after that.

\new
If you are using a special macro in section titles then `\outlines` macro
may crash. You must declare variant of the macro for outlines case which is
expandable using `\regmacro`. See section \ref[toc] for more information
about `\regmacro`.

The command `\insertoutline{<text>}` inserts next entry into PDF outlines at
the main level~0. This entry can be placed before table of contents (created
by `\outlines`) or after it.

\secc Lists 
%%%%%%%%%%

The list of items is surrounded by `\begitems` and `\enditems` commands.
The asterisk (`*`) is active within this environment and it starts one item.
The item style can be chosen by `\style` parameter written after `\begitems`:

\begtt
\style o % small bullet
\style O % big bullet (default)
\style - % hyphen char
\style n % numbered items 1., 2., 3., ...
\style N % numbered items 1), 2), 3), ...
\style i % numbered items (i), (ii), (iii), ...
\style I % numbered items I, II, III, IV, ...
\style a % items of type a), b), c), ...
\style A % items of type A), B), C), ...
\style x % small rectangle
\style X % big rectangle
\endtt

For example:

\begtt
\begitems
* First idea
* Second idea in subitems:
  \begitems \style i
  * First sub-idea
  * Second sub-idea
  * Last sub-idea
  \enditems
* Finito
\enditems
\endtt

produces:

\begitems
* First idea
* Second idea in subitems:
  \begitems \style i
   * First sub-idea
   * Second sub-idea
   * Last sub-idea
  \enditems
* Finito
\enditems

Another style can be defined by the command `\sdef{_item:<style>}{<text>}`.
Default item can be redefined by `\def\normalitem{<text>}`.
The list environments can be nested. Each new level of item is indented by
next multiple of `\iindent` which is set to `\parindent` by default.
The vertical space at begin and end of the environment is inserted by the
macro `\iiskip`.

\secc Tables
%%%%%%%%%%%

The macro `\table{<declaration>}{<data>}` provides similar <declaration>
as in \LaTeX: you can use letters `l`, `r`, `c`, each letter declares 
one column (aligned to left, right, center respectively). 
These letters can be combined by the `|` character (vertical line). Example

\begtt
\table{||lc|r||}{                  \crl
   Month    & commodity & price    \crli \tskip2pt
   January  & notebook   & \$ 700  \cr
   February & skateboard & \$ 100  \cr
   July     & yacht      & k\$ 170 \crl}
\endtt
%
generates the following result:

\bigskip
\hfil\table{||lc|r||}{             \crl
   Month    & commodity & price    \crli \tskip2pt
   January  & notebook   &  \$ 700 \cr
   February & skateboard &  \$ 100 \cr
   July     & yacht      & k\$ 170 \crl}
\bigskip

Apart from `l`, `r`, `c` declarators, you can use the `p{<size>}` declarator
which declares the column of given width. More precisely, a long text in
the table cell is printed as an paragraph with given width.
To avoid problems with narrow left-right aligned paragraphs you can write
`p{<size>\raggedright}`, then the paragraph will be only left aligned.

You can use `(<text>)` in the <declaration>. Then this text is applied in
each line of the table. For example `r(\kern10pt)l` adds more 10\,pt space
between `r` and `l` rows. 

An arbitrary part of the <declaration> can be repeated by a <number>
prefixed. For example `3c` means `ccc` or `c 3{|c}` means
`c|c|c|c`. Note that spaces in the <declaration> are ignored and you 
can use them in order to more legibility.
 
The command `\cr` used in the <data> part of the table (the end row
separator) is generally known. It marks end row of the table. 
Moreover \OpTeX/ defines following similar commands:

\begitems
* `\crl` \dots\ the end of the row with a horizontal line after it.
* `\crli` \dots\ like `\crl` but the horizontal line doesn't intersect the
      vertical double lines.
* `\crlli` \dots\ like `\crli` but horizontal line is doubled.
* `\crlp{<list>}` \dots\ like `\crli` but the lines are drawn only in the
  columns mentioned in comma separated `<list>` of their numbers.
  The <list> can include `<from>-<to>` declarators, for example
  `\crlp{1-3,5}` is equal to `\crlp{1,2,3,5}`. 
\enditems

The `\tskip<dimen>` command works like the `\noalign{\vskip<dimen>}` 
after `\cr*` commands but it doesn't interrupt the vertical lines.

\new
You can use following parameters for the `\table` macro. Default values are listed
too. 

\begtt
\everytable={}       % code used after settings in \vbox before table processing
\thistable={}        % code used when \vbox starts, is is removed after using it
\tabiteml={\enspace} % left material in each column
\tabitemr={\enspace} % right material in each column
\tabstrut={\strut}   % strut which declares lines distance in the table
\tablinespace=2pt    % additional vertical space before/after horizontal lines
\vvkern=1pt          % space between double vertical line
\hhkern=1pt          % space between double horizontal line
\endtt

If you do `\tabiteml={$\enspace}\tabitemr={\enspace$}` then
the `\table` acts like \LaTeX's array environment.

If there is an item which spans to more than one column in the table then
`\multispan{<number>}` macro from plain \TeX{} can help you or, you can use
`\mspan<number>[<declaration>]{<text>}`
which spans <number> columns and formats the <text> by the
<declaration>. The <declaration> must include a declaration of right one column
with the same syntax as common `\table` <declaration>.
If your table includes vertical rules and you want to
create continuous vertical rules by `\mspan`, then use rules
only after `c`, `l` or `r` letter in `\mspan` <declaration>. The
exception is only in the case when `\mspan` includes first
column and the table have rules on the left side. The example of `\mspan` usage is below.

The `\frame{<text>}` makes a frame around <text>. You can put the whole `\table`
into `\frame` if you need double-ruled border of the table. Example:

\begtt
\frame{\table{|c||l||r|}{ \crl
  \mspan3[|c|]{\bf Title} \crl   \noalign{\kern\hhkern}\crli
  first & second & third  \crlli
  seven & eight  & nine   \crli}}
\endtt
%
creates the following result:

%\bigskip
\hfil\frame{\table{|c||l||r|}{\crl
  \mspan3[|c|]{\bf Title} \crl   \noalign{\kern\hhkern}\crli
  first & second & third  \crlli
  seven & eight  & nine   \crli}}
\bigskip

The `c`, `l`, `r` and `p` are default \"<declaration> letters" but you can define
more such letters by `\def\_tabdeclare<letter>{<left>##<right>}`. More about
it is in technical documentation in the file `table.opm`.

The rule width of tables (and implicit width of all `\vrule`s and `\hrule`s)
can be set by the command `\rulewidth=<dimen>`. The default value given 
by \TeX/ is 0.4\,pt.

Many tips about tables can be seen on
\url{http://petr.olsak.net/opmac-tricks-e.html}.

\label[verbatim]\secc Verbatim
%%%%%%%%%%%%%%%%%%%%%%%%%%%%%

The display verbatim text have to be surrounded by the `\begtt` and
`\endtt` couple. 
The in-line verbatim have to be tagged (before and after) 
by a character which is declared by
`\activettchar<char>`. For example \code{\\activettchar`} 
declares the character \code{`} 
for in-line verbatim markup. 
And you can use \code{`\\relax`} for
verbatim `\relax` (for example).
\new
Another alternative of printing in-line
verbatim text is `\code{<text>}` (see below). 

If the numerical register `\ttline` is set to the non-negative value then
display verbatim will number the lines. The first line has the number
`\ttline+1` and when the verbatim ends then the `\ttline` value is equal to the
number of last line printed. Next `\begtt...\endtt` environment will follow
the line numbering. \OpTeX/ sets `\ttline=-1` by default.

The indentation of each line in display verbatim is controlled by
`\ttindent` register. This register is set to the `\parindent` by default.
User can change values of the `\parindent` and `\ttindent` independently.

The `\begtt` command starts internal group in which the catcodes are changed. 
\new
Then the `\everytt` string reister is run. It is empty by default and user can
control fine behavior by it. For example the cactodes can be reset here. If
you need to define active character in the `\everytt`, use `\adef` as in the
following example:

\begtt   \adef@{\string\endtt}
\everytt={\adef!{?}\adef?{!}}
\begtt
Each occurrence of the exclamation mark will be changed to 
the question mark and vice versa. Really? You can try it! 
@
\endtt

The `\adef` command sets its parameter as active {\it after\/}
the body of `\everytt` is read. So you can't worry about active
categories. 

The `\everytt` is applied to all `\begtt...\endtt` environments (if it is not
decared in a group). There are tips for such global `\everytt` definitions here:

\begtt
\everytt={\typosize[9/11]}      % setting font size for verbatim
\everytt={\ttline=0}            % each listing will be numbered from one
\everytt={\adef{ }{\char9251 }} % visualization of spaces (Unicode fonts)
\endtt

\new
If you want to apply a `\everytt` material only for one `\begtt...\endtt`
environment then don't set any `\everytt` but put desired material at the 
same line where `\begtt` is. For example:

\begtt   \adef@{\string\endtt}
\begtt   \adef!{?}\adef?{!}
Each occurrence of ? will be changed to ! and vice versa. 
@
\endtt

The in-line verbatim surrounded by an `\activettchar` doesn't work in
parameter of macros and macro definitions, especially in titles declared by
`\chap`, `\sec` etc. 
\new
You ca use more robust command `\code{<text>}` in such
situations, but you must escape following characters in the <text>:
`\`, `#`, `%`, braces (if the braces are unmatched in the <text>), 
and space or `^` (if there are more than one subsequent spaces or `^` in 
the <text>). Examples:

\begtt
\code{\\text, \%\#} ... prints \text, %#
\code{@{..}*&^$ $}  ... prints @{..}*&^$ $ without escaping, but you can
                        escape these characters too, if you want.
\code{a \ b}        ... two spaces between a  b, the second one must be escaped
\code{xy\{z}        ... xy{z ... unbalanced brace must be escaped
\code{^\^M}         ... prints ^^M, the second ^ must be escaped
\endtt

There is an alternative to `\everytt` named `\everyintt` which is used for
in-line verbatim surrounded by an `\activettchar` or processed by the `\code`
command.

You can print verbatim listing from external files by `\verbinput` command. 
Examples:

\begtt
\verbinput (12-42) program.c  % listing from program.c, only lines 12-42
\verbinput (-60) program.c    % print from begin to the line 60
\verbinput (61-) program.c    % from line 61 to the end
\verbinput (-) program.c      % whole file is printed
\verbinput (70+10) program.c  % from line 70, only 10 lines printed
\verbinput (+10) program.c    % from the last line read, print 10 lines 
\vebrinput (-5+7) program.c   % from the last line read, skip 5, print 7
\verbinput (+) program.c      % from the last line read to the end
\endtt


The `\ttline` influences the line numbering by the same way as in
`\begtt...\endtt` environment. If `\ttline=-1` then real line numbers are
printed (this is default). If \code{\\ttline<-1} then no line 
numbers are printed.

The `\verbinput` can be controlled by `\everytt`, `\ttindent` just like
in `\begtt...\endtt`.


\sec Autogenerated lists
%%%%%%%%%%%%%%%%%%%%%%%%

\secc[toc] Table of contents
%%%%%%%%%%%%%%%%%%%%%%

The `\maketoc` command prints the table of contents of all `\chap`, `\sec`
and `\secc` used in the document. These data are read from external `*.ref` file, so
you have to run \TeX/ more than once (typically three times if the table of
contents is at the beginning of the document). 

The name of the section with table of contents is not printed. The direct usage
of `\chap` or `\sec` isn't recommended here because the table of contents 
is typically not referenced to itself. You can print the unnumbered and unreferenced
title of the section by the code:

\begtt
\nonum\notoc\sec Table of Contents
\endtt

\new
If you are using a special macro in section titles or chapter titles 
and you need different behavior of such macro in other cases then use 
`\regmacro{<case-toc>}{<case-mark>}{<case-outline>}`.
The parameters are applied locally in given cases. The `\regmacro` can be
used repeatedly: the parameters are accumulated (for more macros). 
If a parameter is empty then original definition is used in given case.
For example:

\begtt
% default value of \mylogo macro used in text and in the titles:
\def\mylogo{\leavevmode\hbox{\Red{\it My}\Black{\setfontsize{mag1.5}\rm Lo}Go}}
% another variants:
\regmacro {\def\mylogo{\hbox{\Red My\Black LoGo}}} % used in TOC
          {\def\mylogo{\hbox{{\it My}\/LoGo}}}     % used in running heads
          {\def\mylogo{MyLoGo}}                    % used in outlines
\endtt

\secc Making the index 
%%%%%%%%%%%%%%%%%%%%%

The index can be included into document by `\makeindex` macro. No external
program is needed, the alphabetical sorting are done inside \TeX/ at macro
level.

The `\ii` command (insert to index) declares the word separated by the space
as the index item. This declaration is represented as invisible atom on the
page connected to the next visible word. The page number of the page where
this atom occurs is listed in the index entry. So you can type:

\begtt
The \ii resistor resistor is a passive electrical component ...
\endtt

You cannot double the word if you use the `\iid` instead `\ii`:

\begtt
The \iid resistor is a passive electrical component ...
or:
Now we'll deal with the \iid resistor .
\endtt

Note that the dot or comma have to be separated by space when `\iid` is
used. This space (before dot or comma) is removed by the macro in 
the current text.

The multiple-words entries are commonly organized in the index by the format
(for example): 

\medskip

linear~dependency  11, 40--50

--- independency 12, 42--53

--- space 57, 76

--- subspace 58

\medskip

To do this you have to declare the parts of the words by the `/` separator.
Example:

\begtt
{\bf Definition.}
\ii linear/space,vector/space
{\em Linear space} (or {\em vector space}) is a nonempty set of...
\endtt

The number of the parts of one index entry is unlimited. Note, that you can
spare your typing by the comma in the `\ii` parameter. The previous example
is equivalent to `\ii linear/space` `\ii vector/space`.

Maybe you need to propagate to the index the similar entry to the
linear/space in the form space/linear. You can do this by the shorthand `,@`
at the end of the `\ii` parameter. Example:

\begtt
\ii linear/space,vector/space,@
is equivalent to:
\ii linear/space,vector/space \ii space/linear,space/vector
\endtt

If you really need to insert the space into the index entry, write `~`.

If the `\ii` or `\iid` command is preceded by `\iitype <letter>`, then such
reference (or more references generated by one `\ii`) has specified type.
They should have different format in the index. \OpTeX/ implements 
`\iitype b` and `\iitype i`. This prints the page number in bold or in
italics in the index. Default is empty index type, which prints page numbers
in normal font. See \TeX/book index as an example.

The `\makeindex` creates the list of alphabetically sorted index entries
without the title of the section and without creating more columns. \OpTeX/
provides another macros for more columns: 

\begtt
\begmulti <number of columns>
<text>
\endmulti
\endtt
The columns will be balanced. The Index can be printed by the following
code:

\begtt
\sec Index
\begmulti 3 \makeindex \endmulti
\endtt

Only \"pure words" can be propagated to the index by the `\ii` command. It
means that there cannot be any macro, \TeX/ primitive, math selector etc.
But there is another possibility to create such complex index entry. Use
\"pure equivalent" in the `\ii` parameter and map this equivalent to the
real word which is printed in the index by `\iis` command. Example:

\begtt
The \ii chiquadrat $\chi$-quadrat method is 
...
If the \ii relax `\relax` command is used then \TeX/ is relaxing.
...
\iis chiquadrat {$\chi$-quadrat}
\iis relax {\code{\\relax}}
...
\endtt

The `\iis <equivalent> {<text>}` creates one entry in the {\em dictionary 
of the exceptions}. The sorting is done by the <equivalent> but the
<text> is printed in the index entry list.

The sorting rules when `\makeindex` runs depends on the current language. 
See section~\ref[lang] about lanuages selection.

\secc Bib\TeX/ing
%%%%%%%%%%%%%%%%

The command `\cite[<label>]` or its variants of the type
\hbox{`\cite[<label-1>,<label-2>,...,<label-n>]`}
create the citations in the form [42] or [15,~19,~26]. 
If `\shortcitations` is declared at the beginning of the document then 
continuous sequences of numbers are re-printed like this: 
\hbox{[3--5,~7,~9--11]}. If
`\sortcitations` is declared then numbers generated by one `\cite` command
are sorted upward.

If `\nonumcitations` is used then the marks instead numbers are generated
depending on the used bib-style. For example the citations look like
[Now08] or [Nowak, 2008].

The `\rcite[<labels>]` creates the same list as `\cite[<labels>]` but without
the outer brackets. Example: `[\rcite[tbn], pg.~13]` creates [4,~pg.13].

The `\ecite[<label>]{<text>}` prints the `<text>` only, but the entry labeled
<label> is decided as to be cited. If `\hyperlinks` is used then <text>
is linked to the references list.

You can define alternative formating of `\cite` command. Example:

\begtt
\def\cite[#1]{(\rcite[#1])}    % \cite[<label>] creates (27)
\def\cite[#1]{$^{\rcite[#1]}$} % \cite[<label>] creates^{27}
\endtt

The numbers printed by `\cite` correspond to the same numbers generated in
the list of references. 
There are two possibilities to generate this
references list:

\begitems
* Manually using `\bib[<label>]` commands.
* By `\usebib/<type> (<style>) <bbl-base>` command which reads `*.bib`
  databases directly. 
\enditems

\new
Note that another two possibilities documented in OPmac (using external
Bib\TeX/ program) isn't supported because Bib\TeX/ is old program which does not
supports Unicode. And Biber seems to be not compliant with Plain \TeX.

\medskip\noindent
{\bf References created manually using `\bib[<label>]` command.}

\begtt  
\bib [tbn] P. Olšák. {\it\TeX{}book naruby.} 468~s. Brno: Konvoj, 1997.
\bib [tst] P. Olšák. {\it Typografický systém \TeX.}  
           269~s. Praha: CSTUG, 1995.
\endtt

If you are using `\nonumcitations` then you need to declare the <marks>
used by `\cite` command. To do it you must use long form of the `\bib`
command in the format `\bib[<label>] = {<mark>}`. The spaces around
equal sign are mandatory. Example:

\begtt
\bib [tbn] = {Olšák, 2001} 
    P. Olšák. {\it\TeX{}book naruby.} 468~s. Brno: Konvoj, 2001.
\endtt

\noindent
{\bf Direct reading of `.bib` files} is possible by `\usebib` macro.
This macro reads macro package `opmac-bib.tex` (on demand) which uses the external 
package `librarian.tex` by Paul Isambert. The usage is:

\begtt
\usebib/c (<style>) <bib-base> % sorted by \cite-order (c=cite),
\usebib/s (<style>) <bib-base> % sorted by style (s=style).
% example:
\usebib/s (simple) op-example
\endtt

The <bib-base> is one or more `*.bib` database source files (separated by
spaces and without extension) and the <style> is the part of the filename
`bib-<style>.opm` where the formatting of the references list is
defined. Possible styles are `simple` or `iso690`. The behavior of
`opmac-bib.tex` and `opmac-bib-iso690.tex` is full documented in these files
(after `\endinput` command).

The command `\usebib` select from database files only such bib-records which
were used in `\cite` or `\nocite` commands in your document. It means, not
all records are printed. The `\nocite` behaves as `\cite` but prints
nothing. It only tels that mentioned bib-record should be printed in  
the reference list. If `\notcite[*]` is used then all records from <bib-base>
are printed.

\medskip\noindent
{\bf Formatting of the references list} is controlled by the `\printb` macro.
It is called at the begin of each entry. The default `\printb` macro prints
the number of the entry in square brackets. If the `\nonumcitations` is set
then no numbers are printed, only all lines (but no first one) are indented.
The `\printb` macro can use the following values: `\the\bibnum` (the number
of the entry) and `\the\bibmark` (the mark of the entry used when
`\nonumcitations` is set). Examples:

\begtt
% The numbers are without square brackets:
\def\printbib{\hangindent=\parindent \indent \llap{\the\bibnum. }}
% Printing of <marks> when \nonumcitations is set:
\def\printbib{\hangindent=\parindent \noindent [\the\bibmark]\quad}
\endtt 

Another examples can be found on the 
\ulink[http://petr.olsak.net/opmac-tricks-e.html]{OPmac tricks WWW page}.


\sec Graphics
%%%%%%%%%%%%%

\secc Colors
%%%%%%%%%%%

\OpTeX/ provides a small number of color selectors: 
{\Blue `\Blue`}, 
{\Red `\Red`}, 
{\Brown `\Brown`},
{\Green `\Green`}, 
{\Yellow `\Yellow`}, 
{\Cyan `\Cyan`}, 
{\Magenta `\Magenta`}, 
{`\White`}, 
{\Grey `\Grey`}, 
{\LightGrey `\LightGrey`} and
`\Black`. User can define more
such selectors by setting the CMYK components. For example

\begtt
\def \Orange {\setcmykcolor{0 0.5 1 0}}
\endtt

\new
The command `\morecolors` reads more definitions of color selectors 
from \LaTeX/ file `x11nam.def`. There is about 300 color names like 
`\DeepPink`, `\Chocolate` etc. If there are numbered variants of the same
name, then you can apend letters B, C, etc. to the name in \OpTeX/. For example 
`\Chocolate` is Chocolate1, `\ChocolateB` is Cocolate2 etc.

\new
The color selectors work locally in groups by default but with limitiations. See 
the file `colors.opm` for more information.

Default colors are defined by four CMYK components using `\setcmykcolor` like in
the example above. But you can define a color with three RGB components too by
`\setrgbcolor`, for example `\def\Orange{\setrgbcolor{1 0.5 0}}`. All colors
defined by `\morecolors` are in RGB.

\new
You can define your color by a linear combination of previously defined colors using
`\colordef`. For example:

\begtt
\colordef \myCyan {.3\Green + .5\Blue}  % 30 % green, 50 % blue, rest is white 
\colordef \DarkBlue {\Blue + .4\Black}  % Blue mixed with 40 % of black 
\colordef \myGreen{\Cyan+\Yellow}       % exact the same as \Green
\colordef \MyColor {.3\Orange+.5\Green+.2\Yellow}
\endtt
%
If a convex linear combination (as in the last example above) is used then it
emulates color behavior on a painter's palette. 

Only `\setcmykcolor` is used in default colors in \OpTeX/ and `\colordef`
creates also a colors defined by `\setcmykcolor`. If you define your own
colors by `\setrgbcolor` or you use `\morecolors` then a mix of color spaces
should be in the PDF output. The `\onlyrgb` or `\onlycmyk` commands solves
this problem: only specified color space is used in the
PDF output and if a color is specified in another color space then it is
converted. The `\onlyrgb` creates colors more bright (usable for computer
presentations). On the other hand CMYK makes colors more true when 
they are printing.

More about colors, about CMYK versus RGB and
about the `\colordef` command is written in the `colors.opm` file.

\def\coloron#1#2#3{%
   \setbox0=\hbox{#3}\leavevmode
   {\localcolor\rlap{#1\strut \vrule width\wd0}#2\box0}%
}
The following example defines a macro which creates the
\coloron\Yellow\Brown{colored text on the colored background}. Usage:
`\coloron<background><foreground>{<text>}`

The `\coloron` can be defined as follows:

\begtt
\def\coloron#1#2#3{%
   \setbox0=\hbox{#3}\leavevmode
   {\rlap{#1\strut \vrule width\wd0}#2\box0}%
}
\coloron\Yellow\Brown{The brown text on the yellow backround}
\endtt

{\bf The watermark} is grey text on the backround of the page. \OpTeX/ offers
an example: the macro `\draft` which creates grey scaled and rotated text
DRAFT on the background of every page.

\secc Images
%%%%%%%%%%%

The `\inspic <filename>.<extension><space>` or
`\inspic {<filename>.<extension>}`
inserts the picture stored in
the graphics file  with the name `<filename>.<extension>`. 
You can set the picture width by `\picw=<dimen>`
before `\inspic` command which declares the width of the picture 
The image files can be in the PNG, JPG, JBIG2 or PDF format. 

The `\picwidth` is an equivalent the register to `\picw`. Moreover there is an
`\picheight` register which denotes the height of the picture. If both
registers are set then the picture will be (probably) deformed. 

The image files are searched in `\picdir`. This token string is empty 
by default, this means that the image files are searched in the 
current directory.

If you want to create a vector graphics (diagrams, schema, geometry
skicing) then you can do it in Wysiwyg graphics editor (Inkscape for
example), export the result to PDF and include it by `\inspic`.
If you want to \"proramm" such pictures then Tikz package is recommended.
It works in plain \TeX. 

\secc PDF transformations
%%%%%%%%%%%%%%%%%%%%%%%%

All typesetting elements are transformed by linear
transformation given by the current transformation matrix. The
`\pdfsetmatrix {<a> <b> <c> <d>}` command makes the internal multiplication
with the current matrix so linear transformations can be composed. 
One linear transformation given by the `\pdfsetmatrix` above transforms
the vector $[0,1]$ to $[<a>,<b>]$ and $[1,0]$ to $[<c>,<d>]$.
The stack-oriented commands `\pdfsave` and `\pdfrestore` gives a possibility of
storing and restoring the current transformation matrix and current point.
The possition of current point have to be the same from \TeX{}'s point of
view as from transformation point of view when `\pdfrestore` is processed.
Due to this fact the `\pdfsave\rlap{<transformed text>}\pdfrestore` 
or something similar is recommended.

\OpTeX/ provides two special transformation macros:

\begtt
\pdfscale{<horizontal-factor>}{<vertical-factor>} 
\pdfrotate{<angle-in-degrees>}
\endtt 

These macros simply calls the
properly `\pdfsetmatrix` primitive command.

It is known that the composition of transformations is not commutative. It
means that the order is important. You have to read the transformation
matrices from right to left. Example:

\begtt
First: \pdfsave \pdfrotate{30}\pdfscale{-2}{2}\rlap{text1}\pdfrestore
      % text1 is scaled two times and it is reflected about vertical axis
      % and next it is rotated by 30 degrees left.
second: \pdfsave \pdfscale{-2}{2}\pdfrotate{30}\rlap{text2}\pdfrestore
      % text2 is rotated by 30 degrees left then it is scaled two times
      % and reflected about vertical axis.
third: \pdfsave \pdfrotate{-15.3}\pdfsetmatrix{2 0 1.5 2}\rlap{text3}%
       \pdfrestore % first slanted, then rotated by 15.3 degrees right
\endtt
%
\par\nobreak\bigskip\smallskip
This gives the following result. 
First: \pdfsave \pdfrotate{30}\pdfscale{-2}{2}\rlap{text1}\pdfrestore
second: \pdfsave \pdfscale{-2}{2}\pdfrotate{30}\rlap{text2}\pdfrestore
third: \pdfsave \pdfrotate{-15.3}\pdfsetmatrix{2 0 1.5 2}\rlap{text3}\pdfrestore
\bigskip\bigskip


\sec Others
%%%%%%%%%%%

\secc[lang] Using more languages
%%%%%%%%%%%%%%%%%%%%%%%%%%%%%%%

\OpTeX/ prepares hyphenation patterns for all languages if such patterns are
available in your \TeX/ system. 
\new
Only USenglish patterns (original from Plain \TeX/) are preloaded.
Hyphenation patterns of all another languages are loaded on demand when you first use
the `\<iso-code>lang` command in your document. 
For example `\delang` for German, `\cslang` for
Czech, `\pllang` for Polish. The <iso-code> is a shortcut 
of the language (mostly from ISO 639-1). 
You can list all available languages by `\langlist` 
macro. This macro prints now:

\medskip
{\typosize[9/11.5]\emergencystretch=4em \hbadness=2000
\noindent \langlist
\par}
\medskip

\new
For compatibility with e-plain macros, there is the command
`\uselanguege{<language>}`. The parameter <language> is long form of
language name, i.e.\ `\uselanguage{Czech}` does the same work as `\cslang`.
The `\uselanguage` parameter is case insensitive.

For compatibility with \csplain/ there are macros `\ehyph`, `\chyph`,
`\shyph` which are equivalent to `\enlang`, `\cslang` and `\sklang`.

You can switch between language patterns by `\<iso-code>lang` commands mentioned
above. Defalut is `\enlang`.

\OpTeX/ generates three words used for captions and titles in technical
articles or books: \"Chapter", \"Table" and \"Figure". These words need to be know
in used language and it depends on the previsously used language selectors
`\<iso-code>lang`. \OpTeX/ declares these words only for few languages:
\new
Czech, German, Spanish, French, Greek, Italian, Polish, Russian, Slovak and 
English, If you need to use these words in another languages or you want to
auto-generate more words in your macros, then you can declare it by 
`\sdef` commands as shown in the file `languages.opm`.

The `\makeindex` command needs to know the sorting rules used in your language.
\OpTeX/ defines only few language rules for sorting: Czech,
Slovak and English. How to declare sorting rules for more languages are
described in the file `makeindex.opm`.

\iffalse
And you can optionally
define the `\specsortingdata<iso-code>` macro. Example:
%{\emergencystretch=2em\par}

\begtt
\def\sortingdataes {aAäÄáÁ,bB,cCçÇ,^^P^^Q^^R,dD,...,zZ,.}
\def\specsortingdataes {ch:^^P Ch:^^Q CH:^^R}
\endtt

There are groups of letters separated by comma and ended by comma-dot in
the parameter of the macro `\sortingdata<iso-code>`. (In the example above, three dots must
be replaced by real data by the user.) All letters in one group are not
distinguished during first step of sorting (primary sorting). If some items
are equal from this point of view then the secondary sorting is processed
for such items where all mentioned letters are distinguished in the order
given in the macro. 

Sorting algorithm can treat couple of letters (like Dz, Ch etc.) as one letter 
if the parameter of the macro `\specsortingdata<iso-code>` is defined. There is
a space separated list of items in the form `<couple>:<one-token>`. The
replacing from <couple> to <one-token> is done before sorting, so you can
use `<one-token>` in the `\sortingdata<iso-code>` macro. The `<one-token>`
must be something special not used as the letter of the alphabet. The usage of
`^^A`, `^^B` etc. is recommended but you must avoid the `^^I` and `^^M` because
these characters have special catcode.

The macros `\sortingdata<iso-code>` and `\specsortingdata<iso-code>` are
active when the language selector `\<iso-code>lang` is used.

The list of ignored characters for sorting point of view is defined in the
`\setignoredchars` macro. \OpTeX/ defines this macro like:

{\catcode`\<=12
\begtt
\def\setignoredchars{\setlccodes ,.;.?.!.:.'.".|.(.).[.].<.>.=.+.{}{}}
\endtt
}%
It means that comma, semicolon, question mark, \dots, plus mark are treated
as dot and dot is ignored by sorting algorithm. You can redefine this macro,
but you must keep the format, keep `\setlccodes` in the front and `{}{}` in
the end.
\fi

\secc Pre-defined styles
%%%%%%%%%%%%%%%%%%%%%%%

\OpTeX/ defines two style-declaration macros `\report` and `\letter` 
You can use them at the beginning of your document if you are
preparing these types of document and you don't need to create your own
macros.

The `\report` declaration is intended to create reports. It 
sets default font size to 11\,pt and `\parindent` (paragraph indentation) to 1.2\,em.
The `\tit` macro uses smaller font because we assume that \"chapter" level
will be not used in reports. The first page has no page number, but next pages
are numbered (from number~2). Footnotes are numbered from one in whole
document. The macro `\author <authors><end-line>` can be used when 
`\report` is declared. It prints `<authors>` in italics at center of the
line. You can separate authors by `\nl` to more lines.

The `\letter` declaration is intended to create letters. It sets default
font size to 11\,pt and `\parindent` to 0\,pt. It sets half-line space
between paragraphs. The page numbers are not printed. The `\subject` macro
can be used, it prints the word \"Subject:" or \"Věc" (or something else
depending on current language) in bold. Moreover, the `\address` macro
can be used when `\letter` is declared. The usage of the `\address` macro
looks like:

\begtt
\address
  <first line of address>
  <second line of address>
  <etc.>
  <empty line>
\endtt

It means that you need not to use any special mark at the end of lines: end
of lines in the source file are the same as in printed output. The
`\address` macro creates `\vtop` with address lines. The width of such
`\vtop` is equal to the most wide line used in it. So, you can use
`\hfill\address...` in order to put the address box to the right side of the
document. Or you can use `<prefixed text>\address...` to put 
`<prefixed text>` before first line of the address.

Analogical declaration macros `\book` or `\slides` are not prepared. Each
book needs an individual typographical care so you need to create specific
macros for design. And you can find an inspiration of slides in OPmac tricks
\ulink[http://petr.olsak.net/opmac-tricks-e.html\#slidy]{0017 and 0022}.

\secc Lorem ipsum dolor sit
%%%%%%%%%%%%%%%%%%%%%%%%%%%

\new
A designer needs to concentrate to the design of the output and maybe he/she 
needs a material for testing macros. There is the possibility to generate a
neutral text for such experiments. Use `\lorem[<number>]` or
`\lorem[<from>-<to>]`. It prints a paragraph (or paragraphs) with neutral
text. The numbers <number> or <from>, <to> must be in the range 1 to 150
because there are 150 paragraphs with neutral text prepared for you.
The `\lipsum` macro is equivalent to `\lorem`. Example `\lipsum[1-150]`
prints all prepared paragraphs.

\secc The last page
%%%%%%%%%%%%%%%%%%%

The number of the last page (it may be different from number of pages) is
expanded by `\lastpage` macro. It expands to `?` in first \TeX/ run and to
the last page in next \TeX/ runs. 

There is an example for footlines in the format \"current page / last page": 

\begtt
\footline={\hss \fixedrm \folio/\lastpage \hss}
\endtt

\new
The `\lastpage` expands to the last `\folio` which is a decimal
number or roman nummeral (when `\pageno` is negative). If you need to know
total pages used in the document, use `\totalpages` macro. It expands to 
zero (in first \TeX/ run) or to the number of all pages in the document
(in next \TeX/ runs). 

\secc Use \OpTeX/
%%%%%%%%%%%%%%%%%

\new
The command `\useOpTeX` (or `\useoptex`) does nothing in \OpTeX/ but it causes 
an error (undefined control sequence) when another format is used. You can use it as
the first command in your document:

\begtt
\useOpTeX % we are using OpTeX format, no LaTeX 
\endtt


\sec Summary
%%%%%%%%%%%%

\begtt     \typosize[10/12]\adef!{\string\endtt}\adef&{\kern.25em}
\tit Title (terminated by end of line)
\chap Chapter Title (terminated by end of line)
\sec Section Title (terminated by end of line)
\secc Subsection Title (terminated by end of line)

\maketoc         % table of contents generation
\ii item1,item2  % insertion the items to the index
\makeindex       % the index is generated

\label [labname]  % link target location
\ref [labname]    % link to the chapter, section, subsection, equation
\pgref [labname]  % link to the page of the chapter, section, ...

\caption/t  % a numbered table caption
\caption/f  % a numbered caption for the picture
\eqmark     % a numbered equation

\begitems       % start list of the items
\enditems       % end of list of the items
\begtt          % start verbatim text
!          % end verbatim text
\activettchar X % initialization character X for in-text verbatim
\code           % another alternative for in-text verbatim
\verbinput      % verbatim extract from the external file
\begmulti num   % start multicolumn text (num columns)
\endmulti       % end multicolumn text

\cite [labnames]  % refers to the item in the lits of references
\rcite [labnames] % similar to \cite but [] are not printed.
\sortcitations \shortcitations \nonumcitations % cite format
\bib [labname]  % an item in the list of references
\usebib/? (style) bib-base % direct using of .bib file, ? in {s,c}

\fontfam [FamilyName] % selection of font family
\typosize [font-size/baselineskip] % size setting of typesetting
\typoscale [factor-font/factor-baselineskip] % size scaling
\thefontsize [size] \thefontscale [factor]   % current font size

\inspic file.ext    % insert a picture, extensions: jpg, png, pdf
\table {rule}{data} % simple macro for the tables like in LaTeX

\fnote    % footnote (local numbering on each page)
\mnote    % note in the margin (left or right by page number)

\hyperlinks {color-in}{color-out} % PDF links activate as clickable
\outlines {level}   % PDF will have a table of contents in the left tab

\magscale[factor]  % resize typesetting, line/page breaking unchanged
\margins/pg format (left, right, top, bottom)unit % margins setting

\report \letter    % style declaration macros
\endtt


\sec Compatibility with Plain \TeX/
%%%%%%%%%%%%%%%%%%%%%%%%%%%%%%%%%%%

All macros of plain \TeX/ are re-written in \OpTeX/. Common macros should be
work in the same sense as in original plain \TeX. Internal control sequences
\new 
like `\p@` or `\f@@t` are removed and mostly replaced by control sequences
prefixed by `_` (like `\_this`). All primitives and common macros have two
control sequences in prefixed and unprefixed form with the same
meaning. For example `\hbox` is equal to `\_hbox`. 
Internal macros of \OpTeX/ have and use only prefixed form. User should use
unprefixed forms, but prefixed forms are accessible too, because the `_` is
set as a letter category code globally (in macro files and in users document too). User
should re-define unprefixed forms of control sequences with no worries that
something internal will be broken (only the sequence `\par` cannot be
re-defined without internal change of \TeX/ behavior because it is
hard-coded in \TeX/s tokenization processor).

\new
The Latin Modern 8bit fonts instead Computer Modern 7bit fonts are
preloaded in the format, but only few ones. The full family set is ready to
use after the command `\fontfam[LMfonts]` which reads the fonts in OTF
format.

\new
The accents macros like `\'`, `\v` are undefined in \OpTeX/. Use real
letters like á, ř, ž in your source document instead these old accents macros.
If you really want to use them, you can initialize them by `\oldaccents`
command. But we don't recommend it.

\new
The default paper size is not set as letter with 1\,in margins but as a4 with
2.5\,cm margins. You can change it, for example by 
`\margins/1 letter (1,1,1,1)in`. This example sets the classical plain TeX
page layout.

\new
The origin for typographical area is not at top left 1\,in 1\,in coordinates
but at top left paper corner exactly. For example, `\hoffset` includes directly left
margin.

\bye

