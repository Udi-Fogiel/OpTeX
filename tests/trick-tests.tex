%% Testing autoloaded OpTeX tricks

% run optex tricks-tests twice and then look at resulting PDF and compare
% log output with tricks-tests.dlg

\fontfam[lm]

\eoldef\TEST#1{\par
   \edef\tmp{\detokenize{#1}}
   \wlog{^^J-------- Testing: \tmp ----------}
   \medskip
   \line{\hrulefill\ \lower.5ex\hbox{\tt\tmp}\,\hrulefill}
   \nobreak\medskip
}
\raggedbottom

\TEST \makeLOF \makeLOT

List of figures
\makeLOF

List of tables
\makeLOT

in text:

\captionT [labelX]{A short caption} A detail caption \par
\captionF [labelY]{A short caption} A detail caption \par

\TEST \seccc

\iniseccc

\hyperlinks \Green \Green
\outlines0

\maketoc

\chap  My chapter
\sec   Special section
\secc  This is subsection
\seccc New level of sub-subsection
Text.
\seccc Next text
Text.

\secc Normal subsection

\seccc Next subsubsection

\TEST \incrpp \thepp \thepplast

\newcount\mycounter
{
\headline={The number of objects: \thepplast\mycounter{\the\pageno}\hss}

xx: \incrpp\mycounter \thepp\mycounter,
yy: \incrpp\mycounter \thepp\mycounter.
\vfil\break
zz: \incrpp\mycounter \thepp\mycounter,
uu: \incrpp\mycounter \thepp\mycounter,
vv: \incrpp\mycounter \truepage/\thepp\mycounter.

\vfil\break
}

\TEST \tabs \settabs

\settabs 4\columns
\tabs first text & second text & third text & fourth text & other text \cr

\settabs\tabs &textA &moretextB &textC\cr
\tabs a & b & c \cr
\hbox{textA moretextB textC}

\TEST \pstart \pend

text before

\pstart
\lorem[1-3]
\pend

\pstart
\lorem[2]
\pend

text after

\TEST \twoblocks

\twoblocks{
  {\bf The text in English...} \lorem[1-2]
}
{
  {\bf The text in other language...} \lorem[3-4]
}
\twoblocks{
  {\bf Another text in English...} \lorem[5-10]
}
{
  {\bf Another text in other language...} \lorem[15-20]
}

\TEST \framedblocks

\def\printframe #1#2#3#4{\edef\frameslist{\frameslist
    q 1 1 0 rg 1 0 0 RG 
    1 0 0 1 \bp{\hoffset+10bp} \bp{#2+10bp} cm    % origin shifted to the left-bottom
    \_oval{\bp{\_xhsize-20bp}}{\bp{#4-20bp}}{10} B Q }%
}
\framedblocks

\lipsum[1]
\begblock
  \lipsum[9]
  \begblock
    interni \lipsum[13]
  \endblock
\endblock

\TEST \tabnodes \tabpos

\tablebefore
   \pdfliteral{q 1 0 0 RG \icc[1,1] m \icc[2,3] l S \icc[2,1] m \icc[1,3] l S Q}%
   \tabpos\puttext\ishift\ibl[2,2][3pt,7pt]{HELLO}

\thistable{\tabnodes \tablinespace=0pt}
\table{ccc}{
   first & second & third \cr
   A     & B      & C
}

\TEST \tnote

\table{|p{1.5cm}|p{4cm}|}{ % top alignment (default)
 \crl
 d f g fd s f h d d s r df g h jj f d sd f d f g g t f di ejb  &
 ffj\tnote{aga} h hjs hjs chjds js sjd\tnote{uga} fgh fshs gf sfhs fh ss shfs fs
 \crl
}
\tnoteprint

\TEST \vcent \vbot

\table{|p{1.5cm\vcent}|p{4cm\vcent}|}{ % vertically centered paragraphs
 \crl
 d f g fd s f h d d s r df g h jj f d sd f d f g g t f di ejb  &
 ff h hjs hjs chjds js sjd fgh fshs gf sfhs fh ss shfs fs
 \crl
}
\medskip

\table{|p{1.5cm\vbot}|p{4cm\vbot}|}{ % vertically centered paragraphs
 \crl
 d f g fd s f h d d s r df g h jj f d sd f d f g g t f di ejb  &
 ff h hjs hjs chjds js sjd fgh fshs gf sfhs fh ss shfs fs
 \crl
}

\TEST \longtable

\def\data{head1 & head2 & head22 \cr \tskip5pt}
\fornum 1..70 \do {\addto\data{data#1 & data#1 & data#1 \cr}}
\longtable {|c|c|c|}{\data}

\TEST \crtop \crmid \crbot

\thistable={\tabiteml={}\tabitemr={}\rulewidth=.2pt}
\table{l(\quad)lr}{                           \crtop 
  \mspan2[c]{Item}         &                  \crlp{1-2} 
   Animal    & Description & \quad Price (\$) \crmid 
   Gnat      & per gram    &      13.65       \cr 
             & each        &       0.01       \cr 
   Gnu       & stuffed     &      92.50       \cr 
   Emu       & stuffed     &      33.33       \cr 
   Armadillo & frozen      &       8.99       \crbot} 

\TEST \crx

\thistable={\tabiteml={\quad}\tabitemr={\quad}}
\frame{\table{llllll}{\crx
  aa & bb & cc & dd & ee & ff \crx
  gg & hh & ii & jj & kk & ll \crx
  mm & nn & oo & pp & qq & rr \crx
  ss & tt & uu & vv & ww & xx \crx
  ab & cd & ef & gh & ij & kl \crx
  mn & op & qr & st & uv & wx}}

\TEST \colortab

\thistable={\colortab              % colored cells activated
            \tabstrut={\lower4pt\vbox to15pt{}}        % lines dimensions
            \tabskip=2pt \everycr={\noalign{\kern2pt}} % spaces between cells
            \let\cellcolor=\Yellow % default background color
           }
\table{clr}{first item      & second & \Blue \White third item \cr
            next long item  & \Red X & \Green       YES }

\TEST \keystroke

\keystroke{Q} \keystroke{E} \keystroke{R} \keystroke{T} \keystroke{Y}\par
\keystroke{PgUp} \keystroke{Enter}

\TEST \ignorepis

\ignoreinspic

op-ring picture:
\par\nobreak

{\picw=7cm \inspic op-ring.png }

non exiting picture:

{\inspic nonex.png }

\TEST \roundframe

\roundframe [text-width=8cm, aha=ugg, midrule-color=\Brown, midrule-width=3pt] 
            {Title here}
            {More text}

\bigskip

\roundframe [ um ]
            {Title here}
            {More text}

\bigskip

\roundframe {Title here}
            {Text dgd adhkd had dsglj dagjadg fsj csgsd
             gs sgls fsglfs gfsl fglf gfs rtyr rire wrurey.}

\TEST \shadedframe

\shadedframe {
  {\bf Title}:
  rweyu fiw wi ewyur uwr rwu wruwr rwu rwur uw qu fiw wi ewyur uwr rwu
  wruwr rwu rwur uw q fiw wi ewyur uwr rwu wruwr rwu rwur uw qwr.
}
\bigskip

\shadedframe [frame-color=\Red, bg-color=\Blue, text-color=\White, text-width=6cm] {text}

\TEST \fcread

\fcread (simple) op-biblist

aha \fullcite[texbook] a taky \fullcite[Cmej].

\TEST \easylist

\everylist={\ifcase\ilevel\or \style X \or \style x \else \style - \fi}
\begitems \easylist
* Main one
* Main two
** sub-item
*** sub-sub-item
\enditems

\everylist={}
\begitems \easylist \style m \keepstyle
* First proposition.
** Interesting comment.
*** A note on the comment.
*** Another note.
**** By the way...
***** This is a subsub...-proposition.
* Let’s start something new...
** Next
** Other
\enditems

\TEST \style d

\begitems \style d
*{aha} uff \lipsum[3]
*{dalsi} muff \lipsum[4.1]
\enditems

\lipsum[2]

\TEST \ttlineref

\ttline=0
\begtt \ttlineref §
first line
   second special line §[spec]
   third line
\endtt

Note that the line \lref[spec] is very interesting.

\TEST \style m

\begitems \style m \keepstyle  % prints:
* First                        % 1. First
* Second                       % 2. Second
  \begitems
  * Second-A                   %   2.1 Second-A
  * Second-B                   %   2.2 Second-B
  \enditems
* Third                        % 3. Third
\enditems

\TEST \scaleto

\scaleto 7cm{text}

\scaletof 7cm{text}

\TEST \begfile \createfile \endfile

\begfile{empty}
\endfile

\createfile {program.c}
  tady neco je
\endfile

\TEST \beglua \endlua

\beglua
function simplify(a,b)     
  local function gcd(a,b)   
    if b ~= 0 then
        return gcd(b, a % b)
    else
        return math.abs(a)
    end
  end
  t = gcd(a, b)
  tex.print("{"..a//t.."\\over "..b//t.."}")
end
\endlua
\def\simplify#1#2{\directlua{simplify(#1,#2)}}

Can I make \TeX/ reduce a fraction automatically?

For example, I would like the fraction
$$
  {278\,922\over 74\,088}
$$
to be reduced to
$$
  \simplify{278922}{74088}
$$

\TEST \sethours \setseconds \setweekday \hours \minutes \seconds 

\sethours \setseconds
Current time is \the\hours:\Othe\minutes:\Othe\seconds.

\setweekday
Current day in the week is
\ifcase\weekday Sun\or Mon\or Tue\or Wed\or Thu\or Fri\or Sat\fi.

Today is
\ifcase\month\or Jan\or Feb\or Mar\or Apr\or May\or Jun\or Jul\or
   Aug\or Sep\or Nov\or Dec\fi, \the\day, \the\year
\ or in another format: \the\year/\Othe\month/\Othe\day.

\TEST \showpglists

\showpglists 1

\TEST \runsystem

\runsystem{}
\runsystem{ls}

\TEST \cancel

aha\cancel\\{skrt}.

uff \cancel X{skrt}

\TEST \algol

\begtt \ttlineref § \algol
Require: $n ≥ 0$
Ensure: $y = x^n$
   $y ← 1$
   $X ← x$
   $N ← n$
   while $N ≠ 0$ do !aha! uff §[spec]
      if $N$ is even then
         $X_{\rm new} ← X × X$
         $N ← N/2$
      else {$N$ is odd}
         $y ← y × X$
         $N ← N − 1$
      end if
   end while
\endtt

Dasi zajimavost je na radku~\lref[spec].

\bye
